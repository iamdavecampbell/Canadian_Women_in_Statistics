%%%%%%%%%%%%%%%%%%%%%%%%%%%%%%%%%%
%    Canadian Women In Statistics Poster
%    Latex template with modifications from the original created by Nathaniel Johnston
%    http://www.nathanieljohnston.com/2009/08/latex-poster-template/
%
%    Copyright (C) 2022 Dave Campbell,  Jeeheon Kim
%
%    This program is free software: you can redistribute it and/or modify
%    it under the terms of the GNU General Public License as published by
%    the Free Software Foundation, either version 3 of the License, or
%    (at your option) any later version.
%
%    This program is distributed in the hope that it will be useful,
%    but WITHOUT ANY WARRANTY; without even the implied warranty of
 %   MERCHANTABILITY or FITNESS FOR A PARTICULAR PURPOSE.  See the
%    GNU General Public License for more details.

%    You should have received a copy of the GNU General Public License
%    along with this program.  If not, see <https://www.gnu.org/licenses/>.
%%%%%%%%%%%%%%%%%%%%%%%%%%%%%%%%%%

\documentclass[final]{beamer}
\usepackage[scale=1.24,debug]{beamerposter}
\usepackage{graphicx}			% allows us to import images
 
%-----------------------------------------------------------
% Define the color used for text one timeline node
% ** Later integrate it in style file
%-----------------------------------------------------------
\definecolor{nblue}{RGB}{28,130,185}

%-----------------------------------------------------------
% Define the column width and poster size
% To set effective sepwid, onecolwid and twocolwid values, first choose how many columns you want and how much separation you want between columns
% The separation I chose is 0.024 and I want 4 columns
% Then set onecolwid to be (1-(4+1)*0.024)/4 = 0.22
% Set twocolwid to be 2*onecolwid + sepwid = 0.464
%-----------------------------------------------------------

\newlength{\sepwid}
\newlength{\onecolwid}
\newlength{\twocolwid}
\newlength{\threecolwid}
\setlength{\paperwidth}{48in}
\setlength{\paperheight}{56in}
\setlength{\sepwid}{0.024\paperwidth}
\setlength{\onecolwid}{0.22\paperwidth}
\setlength{\twocolwid}{0.464\paperwidth}
\setlength{\threecolwid}{0.708\paperwidth}
\setlength{\topmargin}{-0.5in}
\usetheme{confposter}
\usepackage{exscale}

%-----------------------------------------------------------
% The next part fixes a problem with figure numbering. Thanks Nishan!
% When including a figure in your poster, be sure that the commands are typed in the following order:
% \begin{figure}
% \includegraphics[...]{...}
% \caption{...}
% \end{figure}
% That is, put the \caption after the \includegraphics
%-----------------------------------------------------------

\usecaptiontemplate{
\small
\structure{\insertcaptionname~\insertcaptionnumber:}
\insertcaption}

%-----------------------------------------------------------
% Define colours (see beamerthemeconfposter.sty to change these colour definitions)
%-----------------------------------------------------------

\setbeamercolor{block title}{fg=ngreen,bg=white}
\setbeamercolor{block body}{fg=black,bg=white}
\setbeamercolor{block alerted title}{fg=white,bg=dblue!70}
\setbeamercolor{block alerted body}{fg=black,bg=dblue!10}

%-----------------------------------------------------------
% Name and authors of poster/paper/research
%-----------------------------------------------------------

\title{Les femmes dans les statistiques au Canada}
%\institute{Carleton University}

%-----------------------------------------------------------
% Start the poster itself
%-----------------------------------------------------------
\geometry{papersize={122cm,151cm}}                  % resize the poster NOTE: find a better way
\begin{document}
\begin{frame}[t]
	\begin{columns}[t]							% the [t] option aligns the column's content at the top
		\begin{column}{\sepwid}\end{column}			% empty spacer column
		\begin{column}{\threecolwid}
			\begin{block}{Chronologie des premières femmes dans le domaine de la statistique au Canada}
				\begin{timeline}{1945}{2020}{5cm}{5cm}{75cm}{115cm}
					\small  
					%\entry{1944}{\color{ngreen} Helen M. Walker becomes the first woman president of the American Statistical Association}% not Canadian but likely of interest
					%\entry{1944}{\color{ngreen} Helen M. Walker becomes the first woman president of the American Statistical Association}% not Canadian but likely of interest
					\entry{1945}{\textbf{Agatha Louisa Chapman} contribue au premier sous-comité des Nations unies sur les statistiques du revenu national.}% {Agatha Louisa Chapman} contributes to the first United Nations Sub-Committee on National Income Statistics.} % 
					\entry{1948}{\color{nblue}Déclaration universelle des droits de l'homme.}     %   Universal Declaration of Human Rights
					\entry{1955}{\color{ngreen} Création de la section montréalaise de l'American Statistical Association.}%Formation of the Montreal Chapter of the American Statistical Association.}% see https://www.jstor.org/stable/2684141
					\entry{1956}{\color{nblue} La loi sur la rémunération des femmes a rendu illégale la discrimination salariale fondée sur le sexe.}%The female pay act made wage discrimination based on gender against the law.}
					\entry{1960}{\color{nblue} Les femmes et les hommes des Premières nations se voient accorder le droit de vote quel que soit leur lieu de résidence et sans renoncer à leur statut.}% First Nations women and men granted the right to vote no matter where they live and without giving up their status.} 
					%   https://electionsanddemocracy.ca/voting-rights-through-time-0/brief-history-federal-voting-rights-canada
					\entry{1967}{\color{ngreen} L'Université du Manitoba crée le premier département de statistique du Canada.}%The University of Manitoba forms Canada's first Department of Statistics.}
					\entry{1968}{\color{ngreen} Formation des sections du sud de l'Ontario et d'Ottawa de l'American Statistical Association.}%Formation of the Southern Ontario and Ottawa Chapters of the American Statistical Association.}% see https://www.jstor.org/stable/2684141
					\entry{1969}{\textbf{Isobel Loutit} devient président de la section montréalaise de l'American Society for Quality Control.}%becomes Chair of the Montreal Section of the American Society for Quality Control.} % also probably Canada's first professional statistician https://ssc.ca/en/profile/isobel-loutit-statistician-quality
					\entry{1970}{\color{nblue} Commission royale d'enquête sur le statut de la femme, qui détaille les discriminations et les obstacles auxquels sont confrontées les femmes au Canada.}%Royal Commission on the Status of Women detailing discrimination and barriers faced by women across Canada.}
					\entry{1971}{\color{nblue} Les prestations de maternité ajoutées à l'assurance chômage.}%Maternity benefits added to unemployment insurance.}
					\entry{1972}{\color{ngreen} Lettres patentes délivrées à l'"Association canadienne de science statistique / Statistical Science Association of Canada".}%Letters Patent issued to the “Statistical Science Association of Canada / Association canadienne de science statistique”.}
					\entry{1972}{\color{ngreen} Création de la Revue canadienne de statistique.}%Canadian Journal of Statistics is created.}
					\entry{1972}{\textbf{Dr. Sylvia Ostry} devient statisticien en chef à Statistique Canada.}%becomes Chief Statistician at Statistics Canada.} %https://www.statcan.gc.ca/eng/blog/stories/sylviaostry
					\entry{1974}{\color{ngreen} Formation de la Société canadienne de statistique.}%Formation of the Canadian Statistical Society.}% see https://www.jstor.org/stable/2684141
					\entry{1975}{\textbf{Dr. Audrey Duthie} et \textbf{Dr. Kathleen Kocherlakota} rejoindre le conseil d'administration de l'Association canadienne de science statistique.}%join the Board of Directors of the Statistical Science Association of Canada.}
					\entry{1977}{\color{nblue} Adoption de la Loi canadienne sur les droits de la personne, qui interdit la discrimination fondée sur le sexe et garantit un salaire égal pour un travail de valeur égale.}%The Canadian Human Rights Act was passed, prohibiting discrimination on the basis of sex, and ensuring equal pay for work of equal value.}
					\entry{1977}{\color{ngreen}Fusion de la Société statistique canadienne et de l'Association des sciences statistiques du Canada au sein de la Société statistique du Canada.}%Merger of the Canadian Statistical Society and the Statistical Science Association of Canada into the Statistical Society of Canada.}
					\entry{1978}{\color{nblue}Le code du travail canadien a été modifié afin d'éliminer la grossesse comme motif de mise à pied ou de licenciement.}%The Canada Labour Code was amended to eliminate pregnancy as a basis for lay-off or dismissal.}
					\entry{1980}{\textbf{Dr. Estella Bee Dagum} reçoit le prix inaugural du Julius Shiskin Memorial Award for Economic Statistics, décerné par la section des statistiques économiques et commerciales de l'American Statistical Association..}%receives the inaugural award of the Julius Shiskin Memorial Award for Economic Statistics, given by the Business and Economic Statistics Section of the American Statistical Association.}
					\entry{1983}{\color{nblue} La Loi canadienne sur les droits de la personne est modifiée pour interdire le harcèlement sexuel et la discrimination fondée sur la grossesse et la situation familiale ou matrimoniale.}% Canadian Human Rights Act was amended to prohibit sexual harassment and to ban discrimination on the basis of pregnancy and family or marital status.}
					\entry{1986}{\color{nblue} La loi fédérale sur l'équité en matière d'emploi a été introduite afin de remédier à la discrimination historique et systémique dont sont victimes les groupes cibles.}%The federal Employment Equity Act was introduced, aimed at redressing historic and systemic discrimination of target group populations.}
					\entry{1986}{\textbf{Nicole P.-Gendreau} devient le premier rédacteur en chef du bulletin d'information SSC Liaison}%Gendreau} becomes inaugural editor of the SSC Liaison newsletter.} %  Nicole Gendreau, Directrice générale, Bureau de la statistique du Québec was also public relations officer: https://ssc.ca/sites/default/files/liaison/liaison-1-1.pdf
					\entry{1986}{\textbf{Dr. Priscilla E. (Cindy) Greenwood} reçoit une subvention de 500 000 dollars de l'Institut d'études avancées Peter Wall.}%Greenwood} receives a grant of \$500,000 from the Peter Wall Institute of Advanced Studies.}
					\entry{1987}{\textbf{Dr. Maureen Tingley} reçoit le prix Pierre Robillard pour sa thèse de doctorat.}%Tingley} is recognized with the Pierre Robillard award for her PhD thesis.}
					\entry{1989}{\textbf{Dr. Judy-Anne Chapman} est à l'origine de la création de la section "Biostatistique", la première section du CSD.}%Chapman} spearheads and leads the formation of the Biostatistics Section, the first Section of the SSC}% see also: https://ssc.ca/en/about/sections-regions/biostatistics/history
					\entry{1990}{\textbf{Dr. Constance van Eeden} reçoit la médaille d'or de la Société statistique du Canada.}%van Eeden} is awarded the Gold medal of the Statistical Society of Canada.}
					\entry{1991}{\textbf{Dr. Agnes Herzberg} devient président de la Société statistique du Canada}%Herzberg} becomes President of the Statistical Society of Canada.}
					\entry{1992}{\textbf{Dr. Brenda MacGibbon} devient président du Comité de sélection des subventions en sciences statistiques du CRSNG.}%MacGibbon} becomes Chair of the NSERC Statistical Sciences Grant Selection Committee.}
					\entry{1992}{\textbf{Dr. Nancy  Reid} reçoit le prix des présidents du COPSS (Committee of Presidents of Statistical Societies) pour ses contributions exceptionnelles à la profession.}%Reid} is awarded the COPSS Presidents' Award by the Committee of Presidents of Statistical Societies for outstanding contributions to the profession.}%
					\entry{1992}{\textbf{Dr. Lise Manchester} reçoit le prix du meilleur article de la Revue canadienne de statistique pour "A technique for comparing graphical methods".}%Manchester} is awarded the Canadian Journal of Statistics award for best paper for  "A technique for comparing graphical methods".}% ref: CJS/RCS vol. 19, 1991, pp 1-22.

					\entry{1997}{\color{ngreen}La Société statistique du Canada crée son comité des femmes en statistique.}%The SSC establishes its Women in Statistics Committee.}
					\entry{1999}{\textbf{Dr. Agnes Herzberg} reçoit le Prix Pour Services Insignes.}%receives SSC Distinguished service award.}
					\entry{2001}{\textbf{Dr. Colleen Cutler} reçoit le prix CRM-SSC en reconnaissance de ses réalisations professionnelles dans le domaine de la recherche au cours des quinze premières années suivant l'obtention de son doctorat.}%Cutler} is awarded the CRM-SSC Prize in recognition of professional accomplishments in research during the first fifteen years after earning a doctorate.}
					\entry{2001}{\textbf{Dr. Charmaine Dean} devient le directeur fondateur du département des statistiques et des sciences actuarielles de l'université Simon Fraser.}%Dean} becomes the Founding Chair of the Department of Statistics and Actuarial Science at Simon Fraser University.}
					\entry{2002}{\textbf{Dr. Jane F. Gentleman} reçoit le prix Janet L. Norwood pour les réalisations exceptionnelles d'une femme dans le domaine des sciences statistiques.}%Gentleman} receives Janet L. Norwood Award for Outstanding Achievement by a Woman in the Statistical Sciences.}
					
					\entry{2002}{\color{ngreen} Lancement de l'Institut national sur les structures de données complexes.}% The National Program on Complex Data Structures launches.}
					\entry{2003}{\textbf{Dr. Nadia Ghazzali} devient vice-recteur à la recherche de l'Université Laval.}%Ghazzali} becomes Deputy Vice-President of Research at Université Laval.}%French: Vice-rectrice adjointe et adjointe au vice-recteur à la recherche
					\entry{2003}{\textbf{Dr. Judy-Anne Chapman} devient le président inaugural du comité d'accréditation de la SSC.}
					\entry{2004}{\textbf{Dr. Barbara Keyfitz} devient directeur de l'Institut Fields.}%Keyfitz} becomes Director of the Fields Institute.}
					\entry{2007}{\textbf{Dr. Agnes Herzberg} reçoit le titre de membre honoraire de la SSC pour ses contributions exceptionnelles au développement de la discipline.}%Herzberg} is awarded SSC Honorary Membership for exceptional contributions to the development of the discipline.}
					\entry{2007}{\color{ngreen}Le prix \textbf{Lise Manchester Award} a été créé pour reconnaîttre l’excellence des travaux statistiques de pointe sur des problèmes d’intérêt public et qui sont potentiellement utiles pour les politiques publiques canadiennes.}%The \textbf{Lise Manchester Award} was established, recognizing  excellence in state-of-the-art statistical work on problems of public interest with potential use in Canadian public policy.}
					\entry{2008}{\color{nblue} Le Premier ministre Stephen Harper présente des excuses aux anciens élèves des pensionnats autochtones canadiens pour les préjudices causés par les objectifs assimilationnistes, les mauvais traitements, et la perte de culture.}%Prime Minister Stephen Harper issues a statement of apology to former students of Residential Schools in Canada for the harm caused by assimilationist goals, abuse, and cultural loss.}
					%% <-- are they "anciens élèves des pensionnats autochtones canadiens"
					%\entry{2008}{\color{ngreen} L'Institut national sur les structures de données complexes met fin à une période de 5 ans extraordinairement fructueuse.}%The National Program on Complex Data Structures ends an extraordinarily successful 5 years.}
					\entry{2009}{\color{ngreen} L'Institut national sur les structures de données complexes obtiennent un financement pour une dernière année.}%The National Institute for Complex Data Structures obtains funding for one year.}
					\entry{2009}{\textbf{Dr. Gail Ivanoff} devient président du groupe d'évaluation des sciences mathématiques et statistiques au CRSNG.}%Ivanoff} becomes Group Chair of the Mathematical and Statistical Sciences Evaluation Group at NSERC.}
					\entry{2012}{\color{ngreen}Lancement de l'Institut canadien de la science statistique.}%Canadian Statistical Science Institute (CANSSI) is launched.}	            
					\entry{2012}{\textbf{Dr. Mary Thompson} devient le premier directeur scientifique de l'Institut canadien de la science statistique.}%Thompson} becomes the inaugural Scientific Director of Canadian Statistical Science Institute (CANSSI).}
					\entry{2012}{\textbf{Dr. Nadia Ghazzali} devient président de l'Université du Québec à Trois-Rivières.}%Ghazzali} becomes President of Université du Québec à Trois-Rivières.}
					\entry{2014}{\textbf{Dr. Nancy Reid} reçoit l'Ordre du Canada.}%Reid} receives the order of Canada.}
					\entry{2016}{\color{nblue} L'Enquête nationale sur les femmes et les filles autochtones disparues et assassinées est lancée en réponse aux appels à l'action lancés par les familles, les communautés et les organisations.}%The  Inquiry into Missing and Murdered Indigenous Women and Girls is launched in response to calls for action from families, communities, and organizations.}

     \entry{2018}{\textbf{Dr. Sevgui Erman} fonde l'accélérateur de science des données à Statistique Canada.}%founds the Data Science Accelerator at Statistics Canada.}
     					\entry{2020}{\textbf{Dr. Maryam Haghighi} devient le directeur fondateur de Data Science à la Banque du Canada.}%Haghighi} becomes the founding Director of Data Science at the Bank of Canada}
					\entry{2020}{\color{ngreen}La SSC crée son comité sur l'équité, la diversité et l'inclusion.}%The SSC establishes its Committee on Equity, Diversity and Inclusion.}					
				\end{timeline}
			\end{block}
		\end{column}
		\begin{column}{\sepwid}\end{column}			% empty spacer column
		\begin{column}{\onecolwid}
			\begin{block}{Acknowledgements}
				This project was supported by 
				\begin{center}
					\includegraphics[width=5in]{SSC.png}
					\includegraphics[width=7in]{carleton.jpg}
					\includegraphics[width=7in]{CANSSI_Logo-tag-stacked.jpg}
				\end{center}
			\end{block}
		\end{column}
		\begin{column}{\sepwid}\end{column}			% empty spacer column
	\end{columns}
\end{frame}
\end{document}

