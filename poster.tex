\documentclass[10pt]{article}

\usepackage[paperwidth=210mm,%
    paperheight=297mm,%
    tmargin=7.5mm,%
    rmargin=7.5mm,%
    bmargin=7.5mm,%
    lmargin=7.5mm,
    vscale=0.5,%
    hscale=0.5]{geometry}

\usepackage[utf8]{inputenc}
\usepackage[T1]{fontenc}
\usepackage{fourier, heuristica} %font

\usepackage{tikz}

\usetikzlibrary{arrows, calc, decorations.markings, positioning}

\pagestyle{empty}

\makeatletter
\newenvironment{timeline}[6]{

    \newcommand{\startyear}{#1}
    \newcommand{\tlendyear}{#2}

    \newcommand{\yearcolumnwidth}{#3}
    \newcommand{\rulecolumnwidth}{#4}
    \newcommand{\entrycolumnwidth}{#5}
    \newcommand{\timelineheight}{#6}

    \newcommand{\templength}{}

    \newcommand{\entrycounter}{0}

    % https://tex.stackexchange.com/questions/85528/checking-whether-or-not-a-node-has-been-previously-defined
    % https://tex.stackexchange.com/questions/37709/how-can-i-know-if-a-node-is-already-defined
    \long\def\ifnodedefined##1##2##3{%
        \@ifundefined{pgf@sh@ns@##1}{##3}{##2}%
    }

    \newcommand{\ifnodeundefined}[2]{%
        \ifnodedefined{##1}{}{##2}
    }

    \newcommand{\drawtimeline}{%
        \draw[timelinerule] (\yearcolumnwidth+5pt, 0pt) -- (\yearcolumnwidth+5pt, -\timelineheight);
        \draw (\yearcolumnwidth+0pt, -10pt) -- (\yearcolumnwidth+10pt, -10pt);
        \draw (\yearcolumnwidth+0pt, -\timelineheight+15pt) -- (\yearcolumnwidth+10pt, -\timelineheight+15pt);

        \pgfmathsetlengthmacro{\templength}{neg(add(multiply(subtract(\startyear, \startyear), divide(subtract(\timelineheight, 25), subtract(\tlendyear, \startyear))), 10))}
        \node[year] (year-\startyear) at (\yearcolumnwidth, \templength) {\startyear};

        \pgfmathsetlengthmacro{\templength}{neg(add(multiply(subtract(\tlendyear, \startyear), divide(subtract(\timelineheight, 25), subtract(\tlendyear, \startyear))), 10))}
        \node[year] (year-\tlendyear) at (\yearcolumnwidth, \templength) {\tlendyear};
    }

    \newcommand{\entry}[2]{%
        % #1 is the year
        % #2 is the entry text

        \pgfmathtruncatemacro{\lastentrycount}{\entrycounter}
        \pgfmathtruncatemacro{\entrycounter}{\entrycounter + 1}

        \ifdim \lastentrycount pt > 0 pt%
            \node[entry] (entry-\entrycounter) [below of=entry-\lastentrycount] {##2};
        \else%
            \pgfmathsetlengthmacro{\templength}{neg(add(multiply(subtract(\startyear, \startyear), divide(subtract(\timelineheight, 25), subtract(\tlendyear, \startyear))), 10))}
            \node[entry] (entry-\entrycounter) at (\yearcolumnwidth+\rulecolumnwidth+10pt, \templength) {##2};
        \fi

        \ifnodeundefined{year-##1}{%
            \pgfmathsetlengthmacro{\templength}{neg(add(multiply(subtract(##1, \startyear), divide(subtract(\timelineheight, 25), subtract(\tlendyear, \startyear))), 10))}
            \draw (\yearcolumnwidth+2.5pt, \templength) -- (\yearcolumnwidth+7.5pt, \templength);
            \node[year] (year-##1) at (\yearcolumnwidth, \templength) {##1};
        }

        \draw ($(year-##1.east)+(2.5pt, 0pt)$) -- ($(year-##1.east)+(7.5pt, 0pt)$) -- ($(entry-\entrycounter.west)-(5pt,0)$) -- (entry-\entrycounter.west);
    }

    \newcommand{\plainentry}[2]{% plainentry won't print date in the timeline
        % #1 is the year
        % #2 is the entry text

        \pgfmathtruncatemacro{\lastentrycount}{\entrycounter}
        \pgfmathtruncatemacro{\entrycounter}{\entrycounter + 1}

        \ifdim \lastentrycount pt > 0 pt%
            \node[entry] (entry-\entrycounter) [below of=entry-\lastentrycount] {##2};
        \else%
            \pgfmathsetlengthmacro{\templength}{neg(add(multiply(subtract(\startyear, \startyear), divide(subtract(\timelineheight, 25), subtract(\tlendyear, \startyear))), 10))}
            \node[entry] (entry-\entrycounter) at (\yearcolumnwidth+\rulecolumnwidth+10pt, \templength) {##2};
        \fi

        \ifnodeundefined{invisible-year-##1}{%
            \pgfmathsetlengthmacro{\templength}{neg(add(multiply(subtract(##1, \startyear), divide(subtract(\timelineheight, 25), subtract(\tlendyear, \startyear))), 10))}
            \draw (\yearcolumnwidth+2.5pt, \templength) -- (\yearcolumnwidth+7.5pt, \templength);
            \node[year] (invisible-year-##1) at (\yearcolumnwidth, \templength) {};
        }

        \draw ($(invisible-year-##1.east)+(2.5pt, 0pt)$) -- ($(invisible-year-##1.east)+(7.5pt, 0pt)$) -- ($(entry-\entrycounter.west)-(5pt,0)$) -- (entry-\entrycounter.west);
    }

    \begin{tikzpicture}
        \tikzstyle{entry} = [%
            align=left,%
            text width=\entrycolumnwidth,%
            node distance=10mm,%
            anchor=west]
        \tikzstyle{year} = [anchor=east]
        \tikzstyle{timelinerule} = [%
            draw,%
            decoration={markings, mark=at position 1 with {\arrow[scale=1.5]{latex'}}},%
            postaction={decorate},%
            shorten >=0.4pt]

        \drawtimeline
}
{
    \end{tikzpicture}
    \let\startyear\@undefined
    \let\tlendyear\@undefined
    \let\yearcolumnwidth\@undefined
    \let\rulecolumnwidth\@undefined
    \let\entrycolumnwidth\@undefined
    \let\timelineheight\@undefined
    \let\entrycounter\@undefined
    \let\ifnodedefined\@undefined
    \let\ifnodeundefined\@undefined
    \let\drawtimeline\@undefined
    \let\entry\@undefined
}
\makeatother


\begin{document}

%\begin{timeline}{1950}{2021}{2cm}{4cm}{10cm}{18cm}
\begin{timeline}{1950}{2021}{2cm}{4cm}{13cm}{28cm}
\entry{1956}{The female pay act made wage discrimination based on gender against the law}
\entry{1967}{The University of Manitoba forms Canada's first Department of Statistics}
\entry{1969}{Isobel Loutit became chair of the Montreal Section of the American Society for Quality Control (ASQC)}
\entry{1970}{Royal Commission on the Status of Women detailing discrimination and barriers faced by women across Canada}
\entry{1971}{Maternity benefits added to unemployment insurance.}
\entry{1972}{Dr. Sylvia Ostry  is the first female Chief Statistician of Canada}
\entry{1975}{Audrey Duthie and Kathleen Kocherlakota are the first women to be elected to the Board of Directors of the Statistical Science Association of Canada, a predecessor to the Statistical Society of Canada (SSC)}
\entry{1977}{The Canadian Human Rights Act (CHRA) was passed, prohibiting discrimination on the basis of sex, and ensuring equal pay for work of equal value.}
\plainentry{1977}{Dr. Priscila E. (Cindy) Greenwood is awarded a grant of \$50,000 from the Peter Wall Institute of Advanced Studies for their first topic of study}
\entry{1978}{The Canada Labour Code was amended to eliminate pregnancy as a basis for lay-off or dismissal.}
\entry{1980}{Estella Bee Dagum receives the inaugural award of the Julius Shiskin Memorial Award for Economic Statistics, given by the Business and Economic Statistics Section of the American Statistical Association.}
\entry{1983}{The CHRA was amended to prohibit sexual harassment and to ban discrimination on the basis of pregnancy and family or marital status.}
\entry{1986}{The federal Employment Equity Act was introduced, aimed at redressing historic and systemic discrimination of target group populations.}
\entry{1990}{Dr. Constance van Eeden is the first woman to receive the Gold medal of the Statistical Society of Canada.}
\entry{1991}{Dr. Agnes Herzberg is the first female president of the Statistical Society of Canada.}
\entry{1992}{Nancy Reid is the first statistician working in Canada to receive the COPSS Presidents' Award.}
\entry{1992}{Lise Manchester is awarded the Canadian Journal of Statistics award or best paper.}
\entry{1993}{K Brenda MacGibbon Taylor is the first woman to be the Chair of the Statistical Sciences Grant Selection Committee.}
\entry{1999}{Dr. Agnes Herzberg is the first woman recipient of the SSC Distinguished service award.}
\entry{2001}{Colleen Cutler is the first woman to be awarded the CRM-SCC Prize.}
\entry{2002}{Jane F. Gentleman wins the first annual Janet L. Norwood Award for Outstanding Achievement by a Woman in the Statistical Sciences.}
\entry{2007}{Agnes Herzberg is the first female to be presented with an SSC Honorary award.}
\entry{2009}{Gail Ivanoff is the first female to be presented with an SSC Honorary award.}
\entry{2012}{Mary Thompson is the first scientific director of Canadian Statistical Science Institute CANSSI.}
\entry{2014}{Nancy Reid became the first female Canadian statistician to receive the order of Canada.}
\entry{2017}{Charmaine Dean became the first female statistician to become Vice President research of a Canadian university.}
\end{timeline}

\end{document}